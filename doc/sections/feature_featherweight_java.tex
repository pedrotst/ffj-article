%\section{Feature Featherweight Java}\label{seq:ffj}

\subsection{Syntax}
The Syntax of \ac{FFJ} is a straightfoward extension of \ac{FJ}. Due to the lack
of space we will not present the formal definition \ac{FJ}, but instead we follow
the same scheme of the \ac{FFJ} original definition in \cite{apel_feature_2008} and present
the modified rules highlited with \hlmod{shaded yellow boxes} and new rules highlighted by
\hlnew{shaded purple boxes}.


The abstract syntax of FJ class declarations, constructor declarations, method
declarations and expressions is given at Table~\ref{abstractsyntax}.

%\begin{center}
\begin{table}[ht!]
    \begin{tabularx}{.62\textwidth}{l|}
            \texttt{CD}~::= \hfill \textit{class declarations:}\\
            \quad \texttt{class\ C~extends~C\ \{\={C} \={f}; K \={M}\}} \\  \\
            \hlnew{\texttt{CR}~::=} \hfill \textit{class refinements:}\\
            \quad \hlnew{\texttt{refines~class C \{\={C} \={f}; KD \={M}\}}} \\ \\
            \texttt{KD}~::=  \hfill\textit{constructor declarations:}\\
            \quad \texttt{C(\={C}~\={f})\{super(\={f});~this.\={f}=\={f};\}}\\\\
            \hlnew{\texttt{KR}~::=} \hfill\textit{constructor refinements:} \\
            \quad \hlnew{\texttt{refines~C(\={E}~\={h}, \={C} \={f})\{original(\={f}); this.\={f}=\={f};\}}} \\\\
            \texttt{MD}~::= \hfill\textit{method declarations:}\\
            \quad \texttt{C~m~(\={C}~\={x})\ \{return~e;\}}
    \end{tabularx}
    \begin{tabularx}{.4\textwidth}{l}
            \hlnew{\texttt{MR}~::=} \hfill \textit{method refinements:}\\
            \quad \hlnew{\texttt{refines C~m~(\={C}~\={x}) \{return~e;\}}}\\ \\
            \texttt{e}~::= \hfill \textit{terms:}\\
            \quad \texttt{x} \hfill\textit{variable}\\ 
            \quad \texttt{e.f} \hfill\textit{field access}\\
            \quad \texttt{e.m(\={e})} \hfill\textit{method invocation}\\
            \quad \texttt{new~C(\={e})} \hfill\textit{object creation}\\
            \quad \texttt{(C)e} \hfill\textit{cast}\\ \\
            \texttt{v}~::= \hfill \textit{values:}\\
            \quad \texttt{new~C(\={e})} \hfill\textit{object creation}
    \end{tabularx}
    \quad
    \caption{\ac{FFJ} Syntax}
    \label{abstractsyntax}
\end{table}
%\end{center}

The variable \texttt{this} is assumed to be included in the set of variables, but
it cannot be used as the name of an argument to a method. 

It is written $\bar{f}$ as a shorthand for a possibly empty sequence
$f_{1},\ldots,f_{n}$~(and similary for $\bar{C},\ \bar{x},\
\bar{e}$ etc.) and $\bar{M}$ as a shorthand for
$M_{1}\ldots M_{n}$~(with no commas).

\begin{table}[ht!]
    \raggedright \textit{Predecessor}\\
	\centering
    \begin{tabular}{c}
        \rowcolor{shpurple}
        \inferrule{index~\texttt{R RT} =\ n\\
                  get~(n-1)~\texttt{RT} = \texttt{P}}
        {\textit{pred } \texttt{R} =\texttt{P}}
    \end{tabular}

    \raggedright \textit{Subtyping}\\
	\centering
	\begin{tabular}{c@{\qquad}c@{\qquad}c}
		\inferrule{ }{\texttt{C~<:~C}} & 
		\inferrule{\texttt{C <: D} \qquad \texttt{C <: E}}
		{\texttt{C~<:~E}} &
		\inferrule{\texttt{class~C~extends~D~\{~\ldots~\}}}
		{\texttt{C~<:~D}} \\
	\end{tabular}
    \label{subtyping}
    \qquad\qquad
    \caption{Subtyping and Predecessor Relations}
\end{table}

\begin{table}[ht!]
	\centering
	\def\arraystretch{2.5}
	\begin{tabular}{|c|}
        \hline
		$fields~($\texttt{Object}$)=\bullet$ \\
		\inferrule{class\ C\ extends\ D~\{\bar{C}\ \bar{f};\ K\
		\bar{M}\} \qquad fields~(D)=\bar{D}\ \bar{g}}
		{fields~(C)=\bar{D}\ \bar{g},\ \bar{C}\ \bar{f}}\\
        \hline
	\end{tabular}
    \label{field}
    \quad\quad
    \caption{Field lookup}
\end{table}


\begin{table}[h!]
	\centering
	\def\arraystretch{3}
	\begin{tabular}{|c|}
        \hline
		\inferrule{class\ C\ extends\ D~\{\bar{C}\ \bar{f};\ K\
		\bar{M}\} \qquad B\ m~(\bar{B}\ \bar{x})\{return\ e;\}\in~\bar{M}} {mtype~(m,~C)=\bar{B}\rightarrow~B} \\
		\inferrule{class\ C\ extends\ D~\{\bar{C}\ \bar{f};\ K\
		\bar{M}\} \qquad m\notin~\bar{M}}
		{mtype~(m,~C)=mtype~(m,~D)} \\
        \hline
	\end{tabular}
    \quad
    \label{mtypelookup}
    \caption{Method type lookup}
\end{table}

\begin{table}[h!]
	\centering
	\def\arraystretch{3}
	\begin{tabular}{c}
		\inferrule{class\ C\ extends\ D~\{\bar{C}\ \bar{f};\ K\
		\bar{M}\} \qquad B\ m~(\bar{B}\ \bar{x})\{return\
	e;\}\in~\bar{M}}
		{mbody~(m,~C)=\bar{x}.e} \\

		\inferrule{class\ C\ extends\ D~\{\bar{C}\ \bar{f};\ K\
		\bar{M}\} \qquad m\notin~\bar{M}}
		{mbody~(m,~C)=mbody~(m,~D)} \\
	\end{tabular}
    \label{mbodylookup}
    \quad
    \caption{Method body lookup}
\end{table}


\begin{table}[h!]
	\centering
	\def\arraystretch{3}
    \caption{Expression typing}
	\begin{tabular}{cr}
		$\Gamma \vdash x:\Gamma(x)$& (T-Var)\\

		\inferrule{\Gamma \vdash e_{0}:C_{0}\qquad fields~(C_{0})=\bar{C}\
		\bar{f}}
		{\Gamma \vdash e_{0}.f_{i}:C_{i}} & (T-Field)\\

		\inferrule{\Gamma \vdash e_{0}:C_{0}\qquad
			mtypes~(m,~C_{0})=\bar{D}\rightarrow C\qquad \Gamma \vdash
		\bar{e} : \bar{C} \qquad \bar{C}~<:~\bar{D}}
		{\Gamma \vdash e_{0}.m(\bar{e}):C} & (T-Invk)\\

		\inferrule{fields(C)=\bar{D}\ \bar{f}\qquad \Gamma \vdash
		\bar{e}:\bar{C} \qquad \bar{C}~<:~\bar{D}}
		{\Gamma \vdash new\ C(\bar{e}):C} & (T-New)\\

		\inferrule{\Gamma \vdash e_{0}:D \qquad D~<:~C}
		{\Gamma \vdash (C)~e_{0}: C} & (T-UCast)\\

		\inferrule{\Gamma \vdash e_{0}:D\qquad C~<:~D \qquad C \neq D}
		{\Gamma \vdash (C)~e_{0}:C} & (T-DCast)\\

		\inferrule{\Gamma \vdash e_{0}:D\qquad C~\nless :~D \qquad D~\nless:~C 
		\qquad stupid\ warning}
		{\Gamma \vdash (C)~e_0:C} & (T-SCast)\\

	\end{tabular}
\quad
\label{exptyping}
\end{table}

\begin{table}[h!]
	\centering
	\def\arraystretch{3}
    \caption{Expression computation}
	\begin{tabular}{cr}
		\inferrule{fields~(C) = \bar{C} \bar{f}}
        {(new\ C(\bar{e})).f_i \rightarrow e_i} & (R-Field) \\

		\inferrule{mbody~(m, C) = \bar{x}.e_0}
        {(new\ C~(\bar{e})).m~(\bar{d}) \rightarrow[\bar{d}/\bar{x}, new\ C~(\bar{e})/this]e_0} & (R-Invk)\\
		\inferrule{C<:D}
        {(D)(new\ C~(\bar{e})) \rightarrow new\ C~(\bar{e})} & (R-Cast)\\
	\end{tabular}
\quad
\label{expcomput}
\end{table}

\begin{table}[h!]
	\centering
	\def\arraystretch{3}
    \caption{Congruence}
	\begin{tabular}{cr}
		\inferrule{e_0 \rightarrow e_0'}
        {e_0.f\rightarrow e_0'.f} & (RC-Field) \\
		\inferrule{e_0 \rightarrow e_0'}
        {e_0.m~(\bar{e})\rightarrow e_0'.m~(\bar{e})} & (RC-Invk-Recv) \\
		\inferrule{e_i \rightarrow e_i'}
        {e_0.m~(\dots, e_i, \dots) \rightarrow e_0'.m~(\dots, e_i, \dots)} & (RC-Invk-Arg) \\
		\inferrule{e_i \rightarrow e_i'}
        {new\ C~(\dots, e_i, \dots) \rightarrow new\ C~(\dots, e_i', \dots)} & (RC-New-Arg) \\
		\inferrule{e_0 \rightarrow e_0'}
        {(C)e_0 \rightarrow (C)e_0'} & (RC-Cast) \\

	\end{tabular}
\quad
\label{expcongr}
\end{table}



A class table CT is a mapping from class names $C$ to class declarations
$L$. A program is a pair (CT, $e$) of a class table and an
expression. In FJ, every class has a superclass declared with extends, with
the exception of \texttt{Object}, which is taken as a distinguished class name
whose definition does \textit{not} appear in the class table. The class table
contains the subtype relation between all the classes. From the
Table~\ref{subtyping}, one can infer that subtyping is the reflexive and
transitive closure of the immediate subclass ralation given by the
\texttt{extends} clauses in CT\@. 


The function $fields$ defined at Table~\ref{field} returns a list of all
the fields declared at the current class definition, as well as the fields
declared at the superclass of it. It returns an empty list in the case of
\texttt{Object}.


The method declaration $D\ m~(\bar{C}\ \bar{x})\ \{return\ e;\}$
introduces a method name $m$ with result type D and parameters
$\bar{x}$ of types $\bar{C}$. The body of the method is the single
statement $return e;$. The variables ${x}$ and the special
variable ${this}$ are bound in ${e}$. The rules that define these
functions are ilustrated in Tables~\ref{mtypelookup} and~\ref{mbodylookup}.

he given class table is assumed to satisfy the following conditions:
\begin{itemize}
	\item $ CT~(C)=class\ C\ldots$ for every $C\in dom(CT)$
	\item \texttt{Object}$\notin dom(CT)$
	\item for every class name $C$~(except \texttt{Object}) appearing anywhere
		in CT, we have $C\in dom(CT)$
	\item there are no cycles in the subtype relation induced by CT, i.e., the
		relation <: is antisymmetric
\end{itemize}

\subsection{Typing}

The typing rules for expressions are in Table~\ref{exptyping}. An environment
$\Gamma$ is a finite mapping from variables to types, written $\bar{c}:\bar{C}$.
The typing judgment for expressions has the form $\Gamma \vdash e: C$, read ``in
the environment $\Gamma$, expression $e$ has type $C$''.

\subsection{Computation}
The reduction relation is of ther form $e \rightarrow e'$, read ``expression
$e$ reduces to expression $e'$ in one step'', We write $\rightarrow *$ for the
reflexisive and transitive closure of $\rightarrow$.

The reduction rules are given in~\ref{expcomput}. There are three reduction
rules, one for field acess, one for method invocation, and one for casting.
These were already explained above. We write $[\bar{d}=\bar{x}, e=y]e_0$ for
the result of replacing $x_1$ by $d_1$, $x_2$ by $d_2, \dots, x_n$ by $d_n$, and $y$ by $e$ in
the expression $e_0$.

Notice again that with the absense of side effects, there is no need of stack
or heap for variable binding. 

The reduction rules may be applied at any point in an expression, so we also
need the obvious congruence rules.

