\section{Related Work}\label{seq:related}

Several techniques have been proposed to implement
\emph{high configurable systems}. Some of
them are based on source code annotations, such as
the \emph{well-known} C preprocessor~\cite{stallman:cpp} and Color IDE~\cite{kastner:icse2008}. Others
rely on compositional approaches, such as
Feature-Oriented Programming~\cite{batory-tse2004,batory_feature-oriented_2004},
Delta-Oriented Programming~\cite{schaefer_delta-oriented_2010}, 
and Aspect Oriented Programming~\cite{kiczales:ecoop2001,alves:splc2005}.
Nevertheless, it is important to
note that, in high configurable systems (such as software
product lines), testing and formal verification are considered
challenging tasks, in particular because, in this context,
these activities must deal with a potential huge number of
products and also consider not only source code artifacts,
but also high-level variability assets (such as feature and
configuration models).

In this scenario, several researchers have explored
the use of core-calculus for languages that
support the development of high configurable systems,
including Imperative Featherweight Delta Java~\cite{schaefer:aosd2011}, Feature
Featherweight Java~\cite{apel_feature_2008}, and Lightweight Feature Java~\cite{delaware:fse-2009}.
To the best of our knowledge, the work of Delaware
et al. was the first to mechanize a core calculus
of a language designed for high configurable systems (in
this case, Lightweight Feature Java)~\cite{delaware:fse-2009}. Differently, here in
this paper we explored the \emph{first mechanization of \acf{FFJ}}
which, according to Apel et al., is a calculus that
addresses the essencials aspects of several
existing implementations of feature-oriented programming
languages, including FST Composer and AHEAD~\cite{apel_feature_2008}. 

For the purpose of evolving our \ac{FJ} mechanization
to \ac{FFJ}, we could have explored some of the
design decisions discussed in previous and elaborate works, such as
\emph{Product Line of Theorems}~\cite{delaware:oopsla2011},
\emph{Data Types \`{a} la Carte}~\cite{swierstra_2008},
and \emph{MetaTheory \'{a} la Carte}~\cite{delaware:popl2013}.
However, we faced with
an engineering trade-off here: although the use of such an
infrastructure could improve the reuse between \ac{FJ} and \ac{FFJ}
implementations in Coq, the accidental complexity involved in
these approaches will actually reduce the comprehensibility of our
specifications and probably delay the conclusion of our
implementations. Therefore, in our opinion, there is still a
gap to guarantee proof extensibility of type systems. 
