\section{Conclusion}\label{seq:conclusion}

Testing and proving software product lines (SPL) %Is it necessary to define the acronym again?
properties (such as type safety and safe evolution) usually is
considered a challenging task. This is due to the fact that
it is usually impractical to individually verify properties of all possible 
instances of a software product line as the number of products grow. 
Moreover, the verification process has to consider different artifacts, 
such as feature models, configuration models, and the SPL asset 
base (that is, requirement models, source code, and so on). 
To partially solve this problem, verification approaches often  
assume that the underlying asset base conforms to a set 
of assumptions{\color{red}, which might \ldots}. 
To mitigate this problem, in this paper we 
present an extension (\ac{FFJ+}) of a core 
calculus for feature-oriented programming, named 
\emph{Feature Featherweight Java} (\ac{FFJ}). \ac{FFJ+} was designed 
side by side with its formalization using Coq and, to the best 
of our knowledge, we are the first to mechanize \ac{FFJ}.  
We believe that such a formalization might help future research efforts 
to fully automate the verification of product line properties 
considering all related assets.
