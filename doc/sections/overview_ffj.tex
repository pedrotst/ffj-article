
\subsection{Overview of \ac{FFJ} and \ac{FFJ+}}\label{seq:offj}

\acf{FFJ} is a core calculus for \acf{FOP}, which was built 
upon an extension of \acf{FJ}---a minimal subset of Java. 
In \ac{FFJ}, classes can be added and modified by the 
introduction of a new feature, that is, 
an existing class can be extended by a class refinement. 
A class refinement is declared like a conventional class, though 
preceded by the keyword \texttt{refines}. For example, 
\texttt{refines class C \{$\dots$\}} refers to a class
refinement that
\emph{refines} the class \texttt{C}. The same can be achieved 
for method introduction and modification. Method refinements,
however, override a previous definition of the corresponding 
method.
 
% \ac{FJ} models a minimal subset of Java. \ac{FFJ} extends \ac{FJ} for \ac{FOP}.
To fully mechanize \ac{FFJ}, we had to disambiguate and enhance 
the language to some extent that it deserves the attention of 
formally documenting these changes. 
Even though these changes are significant, we maintain the philosophy of \ac{FFJ}, \ac{FOP}, 
and Stepwise Refinement, as we discuss in Section~\ref{seq:ffj}.
Due to the lack of space we will not provide neither the formal definition 
of \ac{FJ} nor \ac{FFJ}, but we refer the reader to the original \ac{FJ}~\cite{igarashi_featherweight_2001}
and \ac{FFJ}~\cite{apel_feature_2008} formalizations.
%LEOPOLDO: Essa parte de baixo pode ficar repetitiva, já que vai introduzir o exemplo na sec. anterior. Verificar depois se pode remover
In \ac{FFJ}, as well as in \ac{FFJ+}, classes can be added and 
modified by introducing a new feature (as discussed in the previous section).
An existing class can be extended by a class refinement. A class refinement is similar to a class but
preceded by the keyword \texttt{refines}. For example, \texttt{refines class C@feat \{$\dots$\}} refers to a class refinement that
refines the class \texttt{C}. The same can be achieved for method introduction and modification. Methods refinement,
however, will override the previous definition.

A syntactical difference between \ac{FFJ} and \ac{FFJ+} is that, in \ac{FFJ+}, 
the feature notion appears in the abstract syntax tree (AST) of the language.
While the designers of \ac{FFJ} argue that the programmer does not have 
to explicitly state which feature a class or method belongs to, %Certificar de explicar isto na sec. anterior
we favored the approach of stating the feature name in every refinement.
This simplifies the structure of the language formalism and can be 
seen as an information gathered by the parser to build the AST, and thus 
the actual code expressed using the concrete syntax of this language 
might not have these annotations.

In addition, an \ac{FFJ+} program has separate tables for class 
declarations (\textsf{CT}) and class refinements (\textsf{RT}).
We make this distinction to simplify the extension from \ac{FJ} in \texttt{Coq}, since 
we eliminate the need to match whether a class in the table 
is a refinement or a declaration. From this \textsf{RT}, we can retrieve the 
refinement composition order and build the refinement chain of the program, 
which is used to check if features were correctly composed and
do not reference features that have not been introduced yet. 
Notice that we redefine the denotation of \textsf{RT} from \ac{FFJ}.
In the original version, it was used to retrieve the refinement name given a 
refinement declaration. This is no longer necessary in \ac{FFJ+}, since
that information is already encoded in the syntax.

Finally, in the original \ac{FFJ} definition, the lookup functions are 
somewhat circumvoluted. %evitar somewhat - tentar definir precisamente o que se quer dizer
Accordingly, we propose a very different approach
for them, with the aim as been not only as formal and simple as possible, 
but also easy to evolve from our mechanized version of \ac{FJ}. 
To this end, we eliminate the need for reverse field lookup, reverse method lookup, 
and the refinement relation. A formal description with all these changes 
is given in Section~\ref{subsec:lookup}. Note that, we were only 
able to conceive these improvements while formalizing \ac{FFJ+} in 
\texttt{Coq}. 

%%%SUGESTAO LEOPOLDO PARA ULTIMO PARAGRAFO
%Finally, different from the original \ac{FFJ} definition, 
%for the lookup functions, we eliminate the need for reverse field lookup, reverse method lookup, 
%and the refinement relation. 
%This helps on simplifiying the formalization and also facilitates evolution. 
%A formal description with all these changes  is given in Section~\ref{subsec:lookup}. 
%It is important to notice that only during the formalization process we were
%able to conceive these improvements. 
