\section{Overview of Overhaul Feature Featherweight Java}\label{seq:offj}

\ac{FJ} models a minimal subset of Java. \ac{FFJ} extends \ac{FJ} for \ac{FOP}.
To fully mechanize the language we solved ambiguities and enhanced the language enough
to deserve the attention of formally documenting these changes. 
Even though these changes are significant, the philosofy of \ac{FFJ}, \ac{FOP} and Stepwise Refinement are maintained.
Due to the lack of space we will not provide the formal definition of \ac{FJ} nor \ac{FFJ}.

In \ac{FFJ}, as well as in \ac{FFJ+} classes can be added and modified by the introduction of a new feature.
An existing class can be extended by a class refinement. A class refinement is declared like a class but
preceded by the keyword \texttt{refines}. For example, \texttt{refines class C@feat \{$\dots$\}} refers to a class refinement that
refines the class \texttt{C}. The same can be achieved for method introduction and modification. Methods refinement,
however, will override the previous definition.

The biggest difference from \ac{FFJ} to \ac{FFJ+} is that the feature notion appears in the language syntax.
While the designers of \ac{FFJ} argues that the programmer does not have to explicitly state which
feature a class or method belongs to, we favored the approach of stating the feature in the name of every refinement.
This greatly simplifies the structure of the formalism of the language and can be seen as an information gathered
by the parser to build the AST, i.e. the actual code for this language would not have these annotations.

An \ac{FFJ+} program have a table with every class declaration (\textsf{CT}) and another table with every class refinement (\textsf{RT}).
We make this distinction to simplify the extension from \ac{FJ} in \texttt{Coq} since we eliminate the need
to match whether a class in the table is a refination or a declaration. From this \textsf{RT} we can retrieve the composition order
of the refinements and build the refinement chain of the program, which is used to check if features were composed correctly and
does not references features that weren't yet introduced. Notice that we redefine the denotation of \textsf{RT} from \ac{FFJ}
since it was used to retrieve te refinement name given a refinement declaration, this is no longer needed in \ac{FFJ+} since
that information is already encoded in the syntax.

In the definition of \ac{FFJ} the lookup functions are rather sloppy, hence we propose a very different system
for them to be as formal and simple as possible and also as simple possible to extend \ac{FJ}. 
For example, we eliminate the need for reverse field lookup, reverse method lookup and the refinement relation.
A formal description with all these changes will be given in \ref{subsec:lookup}


