\documentclass[runningheads,a4paper]{llncs}
\usepackage[nolist,nohyperlinks]{acronym}

\acrodef{FOP}[FOP]{Feature Oriented Programming}
\acrodef{ITP}[ITP]{Interactive Theorem Prover}
\acrodef{FJ}[FJ]{Featherweight Java}
\acrodef{FFJ}[FFJ]{Feature Featherweight Java}

\usepackage{amssymb}
\setcounter{tocdepth}{3}
\usepackage{graphicx}
\usepackage[misc]{ifsym}

\usepackage{hyperref}
\hypersetup{
    colorlinks=true,
    citecolor=blue,
    linkcolor=blue,
    filecolor=magenta,      
    urlcolor=cyan,
}

\usepackage{url}


\urldef{\mailsa}\path|abreu223@gmail.com|    
\newcommand{\keywords}[1]{\par\addvspace\baselineskip
\noindent\keywordname\enspace\ignorespaces#1}

\begin{document}

\mainmatter  % start of an individual contribution

% first the title is needed
\title{Mechanization and Overhaul of Feature Featherweight Java}

% a short form should be given in case it is too long for the running head
\titlerunning{Mechanization and Overhaul of Feature Featherweight Java}

% the name(s) of the author(s) follow(s) next
%
% NB: Chinese authors should write their first names(s) in front of
% their surnames. This ensures that the names appear correctly in
% the running heads and the author index.
%
\author{Pedro Abreu\textsuperscript{(\Letter)}%
%\thanks{Please note that the LNCS Editorial assumes that all authors have used
%the western naming convention, with given names preceding surnames. This determines
%the structure of the names in the running heads and the author index.}%
\and Rodrigo Bonif\'acio}
%
\authorrunning{Lecture Notes in Computer Science: Authors' Instructions}
% (feature abused for this document to repeat the title also on left hand pages)

% the affiliations are given next; don't give your e-mail address
% unless you accept that it will be published
\institute{Departamento de Ci\^encia da Computa\c{c}\~ao, Universidade de Bras\'{\i}lia, Brazil\\
\mailsa}
%
% NB: a more complex sample for affiliations and the mapping to the
% corresponding authors can be found in the file "llncs.dem"
% (search for the string "\mainmatter" where a contribution starts).
% "llncs.dem" accompanies the document class "llncs.cls".
%

\toctitle{Lecture Notes in Computer Science}
\tocauthor{Authors' Instructions}
\maketitle


\begin{abstract}
Specifying a language in an \ac{ITP} is seldom faithful
to its original pen-and-paper specification. However the process of mechanizing the
language and type safety proofs might unearth insights for clearer overall specification.
In the present we discuss the process of extending a mechanization of \textit{\ac{FJ}}
with \textit{\ac{FFJ}} in \texttt{Coq} and how that lead us to a
clearer, unambiguous and equivalent syntax and semantics while keeping the proofs as simple as possible.
This is also the first mechanization of a \textit{\ac{FOP}} language to date.


\keywords{Language Design, Language Semantics, Java, FOP, Coq}
\end{abstract}

\section{Introduction}
\textit{\acf{FOP}} \cite{prehofer_feature-oriented_1997} is a design methodology and tools for program synthesis \cite{batory_tutorial_2003}.
It aims at the modularization of software systems in terms of features. A \textit{feature}
implements a stakeholder's requirement and is typically an increment in program functionality.
When added to a software system, a feature introduce new structures, such as classes and methods,
and refines existing ones, such as extending methods bodies.

There are several \ac{FOP} languages and tools that provides varying mechanisms
that support the specification and composition of features properly, such as AHEAD \cite{batory_feature-oriented_2004},
FSTComposer \cite{apel_superimposition:_2008}, FeatureC++ \cite{apel_featurec++:_2005}, and more recently Delta-Oriented Programming \cite{schaefer_delta-oriented_2010}. \ac{FOP} has also been used to develop
\textit{product-lines} in disparate domains, including compilers for extensible Java dialects 
\cite{batory_jts:_1998}, fire support simulators for U.S. Army \cite{batory_achieving_2000}, high-performance network
\cite{batory_design_1992}, and program verification tools \cite{kurt_stirewalt_component-based_2001}.

Due to the relevance of \ac{FOP}, falar sobre lps...

Several attempts to formalize the type system of \ac{FOP} languages have been made. %Citar todas formalizações de delta oriented programmaning
For instance,  \textit{\acf{FFJ}} \cite{apel_feature_2008} is a proposed type system for \ac{FOP} languages and tools, 
which is developed on top of \textit{\acf{FJ}} \cite{igarashi_featherweight_2001}
to provide a simple syntax and semantics conforming with common \ac{FOP} languages, 
incorporating constructs for feature composition. %Falar de alguma formalizacao de DOP.

Nevertheless, none of the existing efforts for specifying \ac{FOP} type system have been mechanized to date.
Which means that we have no formal guarantees that the specification is type-safe other than peer review. 
Such a method is known for enabling small errors to remain hidden for several years, specially as the proofs grow larger and harder to follow.
The idea behind mechanization is to check these proofs with the aid of a computer, reducing significantly the risk of errors, while 
taking full use of automation for the tedious or straightforward steps of the proof.

In this paper, we present an implementation of \ac{FJ} which we extend with \ac{FFJ} using \texttt{Coq}.
The process of scrutinizing \ac{FFJ} and defining \textit{unambigously} its semantics in \texttt{Coq} lead us to some language specification and implementation improvements. 
Altogether these improvements makes more natural the transition between \ac{FJ} and \ac{FFJ}, simplifying the auxiliary functions used in the language specification as well as the type safety proofs and lemmas. 
Not only to be easier to increment \ac{FJ} syntax and semantics, but also easier increment its proofs leading to a clearer and simpler specification of \textit{FFJ}.
Hence we can sumarize the main contribution of this paper as follows:
\begin{enumerate}
    \item The first mechanization of a FOP type system
    \item A better specification of FFJ, which may help other researchers to reason about software product lines properties.
    \item A report about the benefits of using a proof assistant 
    to revamp an existing specification of a non-mechanized language type system.
\end{enumerate}

This paper is organized as follows: in the Section~\ref{seq:coq} we give a brief introduction to \texttt{Coq} Section 3 briefly introduces software product lines, \ac{FOP} and \ac{FFJ},
Section 4 formally describes our revamped \ac{FFJ} and states the lemmas needed to preserve \ac{FJ} increment to \ac{FFJ} type safety, Section 5 discuss the implications of these results to the product line research and Section 6 
is the conclusion and shows possible future works.

\section{Coq}\label{seq:coq}
Lorem Ipsum

\bibliography{bibliography}{}
\bibliographystyle{splncs03}



%\begin{thebibliography}{5}
%\end{thebibliography}

%esta parte fica para outra sesssao
%The main idea of \ac{FFJ} is to provide a timeline of increment to \ac{FJ} classes and methods called refinements. With that in hand, job of
%proving that \ac{FFJ} maintain \ac{FJ} type safety comes down to showing that the refinements respects former typing rules. And that job is 
%fairly easy given that \ac{FFJ} syntax does not affect \ac{FJ} typing rules.

\end{document}
